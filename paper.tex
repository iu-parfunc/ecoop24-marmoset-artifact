\documentclass[a4paper,UKenglish, authorcolumns]{darts-v2021}
%This is a template for producing DARTS artifact descriptions. 
%for A4 paper format use option "a4paper", for US-letter use option "letterpaper"
%for british hyphenation rules use option "UKenglish", for american hyphenation rules use option "USenglish"
% for section-numbered lemmas etc., use "numberwithinsect"
%for anonymousing the authors (e.g. for double-blind review), add "anonymous"
%for enabling a two-column layout for the author/affilation part (only applicable for > 6 authors), use "authorcolumns"
%for producing a PDF according the PDF/A standard, add "pdfa"
 
\usepackage{microtype}%if unwanted, comment out or use option "draft"
\usepackage{xspace}


%\graphicspath{{./graphics/}}%helpful if your graphic files are in another directory

%\nolinenumbers to disable line numbers

\bibliographystyle{plainurl}% the mandatory bibstyle

% Commands for artifact descriptions
% Written by Camil Demetrescu and Erik Ernst
% April 8, 2014

\newenvironment{scope}{\section{Scope}}{}
\newenvironment{content}{\section{Content}}{}
\newenvironment{getting}{\section{Getting the artifact} The artifact 
endorsed by the Artifact Evaluation Committee is available free of 
charge on the Dagstuhl Research Online Publication Server (DROPS).}{}
\newenvironment{platforms}{\section{Tested platforms}}{}
\newcommand{\license}[1]{{\section{License}#1}}
\newcommand{\mdsum}[1]{{\section{MD5 sum of the artifact}#1}}
\newcommand{\artifactsize}[1]{{\section{Size of the artifact}#1}}


% Author macros::begin %%%%%%%%%%%%%%%%%%%%%%%%%%%%%%%%%%%%%%%%%%%%%%%%
\title{Optimizing Layout of Recursive Datatypes with Marmoset (Artifact)} %TODO Please add

%\titlerunning{A Sample DARTS Research Description (Artifact)} %optional, in case that the title is too long; the running title should fit into the top page column

% ARTIFACT: Authors may not be exactly the same as the related scholarly paper, e.g., you may want to include authors who contributed to the preparation of the artifact, but not to the companion paper

\author{Vidush Singhal}{Purdue University, USA}
{singhav@purdue.edu}{https://orcid.org/0000-0001-6912-3840}
{}

\author{Chaitanya Koparkar}{Indiana University, USA}
{ckoparka@indiana.edu}{https://orcid.org/0000-0002-4515-8499}
{}

\author{Joseph Zullo}{Purdue University, USA}
{jzullo@purdue.edu}{https://orcid.org/0000-0002-3908-9853}
{}

\author{Artem Pelenitsyn}{Purdue University, USA}
{apelenit@purdue.edu}{https://orcid.org/0000-0001-8334-8106}
{}

\author{Michael Vollmer}{University of Kent, UK}
{M.Vollmer@kent.ac.uk}{https://orcid.org/0000-0002-0496-8268}
{}

\author{Mike Rainey}{Carnegie Mellon University, USA}
{me@mike-rainey.site}{https://orcid.org/0009-0002-9659-1636}
{}

\author{Ryan Newton}{Purdue University, USA}
{rrnewton@purdue.edu}{https://orcid.org/0000-0003-3934-9165}
{}

%$^*$
\author{Milind Kulkarni}{Purdue University, USA}
{milind@purdue.edu}{https://orcid.org/0000-0001-6827-345X}
{}


%\author{John Q. Public}{Dummy University Computing Laboratory, [optional: Address], Country \and My second affiliation, Country \and \url{http://www.myhomepage.edu} }{johnqpublic@dummyuni.org}{https://orcid.org/0000-0002-1825-0097}{(Optional) author-specific funding acknowledgements}%TODO mandatory, please use full name; only 1 author per \author macro; first two parameters are mandatory, other parameters can be empty. Please provide at least the name of the affiliation and the country. The full address is optional

%\author{Joan R. Public\footnote{Optional footnote, e.g. to mark corresponding author}}{Department of Informatics, Dummy College, [optional: Address], Country}{joanrpublic@dummycollege.org}{[orcid]}{[funding]}

%\authorrunning{J.\,Q. Public and J.\,R. Public}%TODO mandatory. First: Use abbreviated first/middle names. Second (only in severe cases): Use first author plus 'et al.'

\authorrunning{V. Singhal et al.}

%\Copyright{John Q. Public and Joan R. Public}%TODO mandatory, please use full first names. LIPIcs license is "CC-BY";  http://creativecommons.org/licenses/by/3.0/

\Copyright{Vidush Singhal, Chaitanya K., Joseph Z., Artem P., Michael V., Mike R., Ryan N. and Milind K.}

%\ccsdesc[100]{General and reference~General literature}
%\ccsdesc[100]{General and reference}%TODO mandatory: Please choose ACM 2012 classifications from https://dl.acm.org/ccs/ccs_flat.cfm 

\ccsdesc[500]{Software and its engineering~Compilers}
\ccsdesc[500]{Software and its engineering~Software performance}
\ccsdesc[500]{Information systems~Data layout}

%\keywords{Dummy keyword}%TODO mandatory; please add comma-separated list of keywords

\keywords{Tree traversals, Compilers, Data layout optimization, Dense data layout} %TODO mandatory; please add comma-separated list of keywords


%TODO Please provide information to the related scholarly article
\RelatedArticle{``Optimizing Layout of Recursive Datatypes with Marmoset''}

\acknowledgements{This work was supported in part by NSF CCF-1908504, CCF-1919197, CCF-2216978, CCF-2119352, CCF-1909862
and EPSRC EP/X021173/1.}
%\funding{(Optional) general funding statement \dots}%optional, to capture a funding statement, which applies to all authors. Please enter author specific funding statements as fifth argument of the \author macro.

%\nolinenumbers %uncomment to disable line numbering

%Editor-only macros:: begin (do not touch as author)%%%%%%%%%%%%%%%%%%%%%%%%%%%%%%%%%%
\Volume{3}
\Issue{2}
\Article{1}
\RelatedConference{42nd Conference on Very Important Topics (CVIT 2016), December 24--27, 2016, Little Whinging, United Kingdom}
% Editor-only macros::end %%%%%%%%%%%%%%%%%%%%%%%%%%%%%%%%%%%%%%%%%%%%%%%

\newcommand{\system}{{\sc Marmoset}\xspace}
\newcommand{\gibbon}{{\sc Gibbon}\xspace}
\newcommand{\mlton}{{\sc MLton}\xspace}

\providecommand{\tightlist}{%
  \setlength{\itemsep}{0pt}\setlength{\parskip}{0pt}}

\begin{document}

\maketitle

\begin{abstract}
    While programmers know that memory representation of data structures can have significant effects
    on performance, compiler support to {\em optimize} the layout of those structures is an under-explored field.
    Prior work has optimized the layout of individual, \textit{non-recursive} structures without considering how
    collections of those objects in linked or \textit{recursive} data structures are laid out.
    %The internal layout of constituent objects in linked data structures is an under-explored field.
    
    %or focused on arranging the
    %placement of linked data structures without considering the internal layout of the constituent objects.
    
    This work introduces \system, a compiler that optimizes the layouts of algebraic datatypes, with a special focus
    on producing highly optimized, {\em packed} data layouts where recursive structures can be traversed with minimal
    pointer chasing. \system performs an analysis of how a recursive ADT is used across functions to choose a \textit{global} layout
    that promotes simple, strided access for that ADT in memory. It does so by building and solving a constraint system
    to minimize an abstract cost model, yielding a predicted efficient layout for the ADT. \system then builds on top
    of \gibbon, a prior compiler for packed, mostly-serial representations, to synthesize optimized ADTs. We show experimentally
    that \system is able to choose optimal layouts across a series of microbenchmarks and case studies, outperforming both
    \gibbon's baseline approach, as well as \mlton, a Standard ML compiler that uses traditional
    pointer-heavy representations.
\end{abstract}

% ARTIFACT: please stick to the structure of 7 sections provided below

% ARTIFACT: section on the scope of the artifact (what claims of the paper are intended to be backed by this artifact?)
\begin{scope}
%What is the scope of the artifact? What claims of the related scholarly paper are intended to be backed by this artifact?

This artifact is an accompaniment to the conference paper "Optimizing Layout of Recursive Datatypes with Marmoset".
This artifact helps validate the claims made in the evaluation section of the paper. 
It helps to validate the results shown in Table 1-5 and Figures 8 and 9. 
\end{scope}

% ARTIFACT: section on the contents of the artifact (code, data, etc.)
\begin{content}
The artifact package includes:
\begin{itemize}
\item A README with instructions on how to install the artifact.
\item A Dockerfile to build a Docker~\cite{merkel2014docker} image. 
\item A tar ball of the pre built docker image.
\item Python scripts to run the experiments.
\item Source files of benchmarks. 
\end{itemize}
\end{content}

% ARTIFACT: section containing links to sites holding the
% latest version of the code/data, if any
\begin{getting}
% leave empty if the artifact is only available on the DROPS server.
% otherwise, provide links to the latest version of the artifact (e.g., on github)
In addition, the artifact is also available at:
\begin{itemize}
\item Zenodo: \url{https://zenodo.org/records/10732462}
\item Github: \url{https://github.com/iu-parfunc/ecoop24-marmoset-artifact.git}
\end{itemize}
For the most current version of the artifact, please check the github repository 
for the most recent updates.
\end{getting}

% ARTIFACT: section specifying the platforms on which the artifact is known to
% work, including requirements beyond the operating system such as large
% amounts of memory or many processor cores
\begin{platforms}
The Artifact will work on any CPU configuration. However, it requires a large RAM for bigger 
datasets to work. The experiments were run on an AMD Ryzen Threadripper 3990X 64-Core processor.
It has 252 Gb of RAM and the operating system is Ubuntu 22.04.3 LTS.
It is recommended to run the experiments on a system with > 100 GB of RAM. 
\end{platforms}

% ARTIFACT: section specifying the license under which the artifact is
% made available
\license{The artifact is available under The Creative Commons Attribution 4.0 International License.~\cite{CC-Liscense}}

% ARTIFACT: section specifying the md5 sum of the artifact master file
% uploaded to the Dagstuhl Research Online Publication Server, enabling 
% downloaders to check that the file is the expected version and suffered 
% no damage during download.
\mdsum{224d1336264a5fd07c2cc56d525ee33e}

% ARTIFACT: section specifying the size of the artifact master file uploaded
% to the Dagstuhl Research Online Publication Server
\artifactsize{3.0GB}

\hypertarget{ecoop-2024-artifact-marmoset}{%
\section{ECOOP 2024 Artifact -- User Guide}\label{ecoop-2024-artifact-marmoset}}

Title of the submitted paper:

\begin{quote}
Optimizing Layout of Recursive Data Types with Marmoset
\end{quote}

\hypertarget{quick-start-guide}{%
\subsection{Quick-start guide}\label{quick-start-guide}}

The artifact is bundled as an OCI container created with Podman
(Dockerfile is available). The Docker image is \texttt{tar}red as
\texttt{marmosetArtifact.tar} The image can be added to the local Docker
store as follows:

\begin{verbatim}
docker load --input marmosetArtifact.tar
\end{verbatim}

Alternatively, you can build the image with the Dockerfile

\begin{verbatim}
DOCKER_BUILDKIT=1 docker image build -t marmoset -f Dockerfile .
\end{verbatim}

Once you get the image, start the session as follows (so called
\texttt{CMD1}):

\begin{verbatim}
docker run --rm -ti marmoset
\end{verbatim}

For the kick-the-tires stage, execute the following commands upon
entering the container:

\begin{verbatim}
cd vsGibbon
./generate_runtimes.py --run quick
\end{verbatim}

This should take about a minute and at the end print the contents
equivalent to the contents of Tables 1 and 2 in the paper. The
\texttt{quick} mode is specifically developed for the kick-the-tires
stage.

\hypertarget{overview}{%
\subsection{Overview}\label{overview}}

\begin{itemize}
\item
  type of artifact --- code, in particular:

  \begin{itemize}
  \tightlist
  \item
    source files of the Gibbon~\cite{gib1, gib2} compiler with our extension to it called
    Marmoset.
  \item
    Gibbon binaries including Marmoset.
  \item
    Python and Bash scripts to run Gibbon, Marmoset and GHC to reproduce
    the main tables and figures of the paper.
  \end{itemize}
\item
  format --- Marmoset is implemented in Haskell (like the rest of
  Gibbon). The scripts produce a combination of \texttt{.csv} and
  \texttt{.pdf} files holding the data in the tables and figures. Also,
  the benchmark programs discussed in the paper are stored as files with
  the \texttt{.hs} extension.
\item
  location in the container --- After entering the container via the
  instructions provided above, the structure of the container is as
  follows. The default working directory and the \texttt{\$HOME} in the
  container is \texttt{/root}. The \texttt{run.sh} script in this
  directory may serve as a master script to run either the
  \texttt{small} set of inputs or the full set (the default). The two
  folders \texttt{\textasciitilde{}/vsGibbon},
  \texttt{\textasciitilde{}/vsGHC} and \texttt{\textasciitilde{}/vsSML}
  contain benchmarks to evaluate Marmoset against Gibbon, GHC and
  standard ML, following the Evaluation section of the paper. The
  \texttt{\textasciitilde{}/marmoset} directory contains the Gibbon
  compiler with the Marmoset extension. The code is pre-built and
  available in \texttt{\$PATH} as \texttt{gibbon} (Marmoset is activated
  by \texttt{gibbon} flags).
\end{itemize}

\hypertarget{for-authors-claiming-an-available-badge}{%
\subsection{Proof of availability}\label{for-authors-claiming-an-available-badge}}

We posted the artifact on Zenodo in addition to Dagstuhl:
\url{https://doi.org/10.5281/zenodo.10578861}. The artifact is available under the
\texttt{Creative\ Commons\ Attribution\ 4.0\ International} license.

\hypertarget{for-authors-claiming-a-functional-or-reusable-badge}{%
\subsection{Proof for a functional or reusable
badge}\label{for-authors-claiming-a-functional-or-reusable-badge}}

\hypertarget{files-structure-in-the-container}{%
\subsubsection{Files structure in the
container}\label{files-structure-in-the-container}}

The source files of Gibbon and Marmoset reside in
\texttt{\textasciitilde{}/marmoset} (\texttt{\textasciitilde{}} is
\texttt{/root}).

All scripts and benchmarks reside in one of the three directories in the
container:

\begin{enumerate}
\def\labelenumi{\arabic{enumi}.}
\item
  \texttt{\textasciitilde{}/vsGibbon} --- evaluation for Gibbon and
  Marmoset (Tables 1--3 and Figure 9).
\item
  \texttt{\textasciitilde{}/vsSML} --- evaluation for Marmoset vs MLton
  (Figure 10).
\item
  \texttt{\textasciitilde{}/vsGHC} --- evaluation for Marmoset vs GHC
  (for the extended version of the paper).
\end{enumerate}

In (1) and (3), there are two subdirectories that contain \texttt{small}
and \texttt{large} benchmarks respectively. The two kinds of benchmark
programs differ only in sizes of inputs. The \texttt{large} variant
should more faithfully reproduce the results in the paper but requires
big RAM (\textgreater100Gb).

Four Python scripts map on the figures and tables in the paper as
follows:

\begin{enumerate}
\def\labelenumi{\arabic{enumi}.}
\item
  \texttt{\textasciitilde{}/vsGibbon/generate\_runtimes.py} ---
  generates the run times for Gibbon, Marmoset-greedy, and
  Marmoset-solver, \texttt{Tables\ 1-3}
\item
  \texttt{\textasciitilde{}/vsGibbon/generate\_compile\_times.py} ---
  generates the compile times, \texttt{Figure\ 9}.
\item
  \texttt{\textasciitilde{}/vsGibbon/generate\_cache\_stats.py} ---
  generates the statistics for CPU cache, \texttt{Table\ 5}.
\item
  \texttt{\textasciitilde{}/vsSML/generate\_sml\_numbers.py} ---
  generates the run times for MLton, \texttt{Figure\ 8}.
\item
  \texttt{\textasciitilde{}/vsGHC/generate\_ghc\_numbers.py} ---
  generates the run times for GHC as presented in the extended version
  of the paper~\cite{singhal2024optimizinglayoutrecursivedatatypes}.
\end{enumerate}

Script (3) relies on the PAPI~\cite{icl:31} framework, which does not work inside a
container. Below we provide instructions for running it outside the
container (section ``Build Marmoset and PAPI outside Docker for
generating Table 5'').

The other three scripts can run sequentially from one master script
called \texttt{\textasciitilde{}/run.sh}. The master script accepts the
\texttt{small} flag, as well as scripts (1) and (5), so that the results
can reproduce at a smaller scale on an average consumer machine using
the benchmarks in the \texttt{small} directories.

After executing \texttt{CMD1} (see above) and entering the container,
use either the master script \texttt{\textasciitilde{}/run.sh} or the
individual scripts to reproduce the figures and tables. For example,
\texttt{./generate\_runtimes.py} (inside \texttt{vsGibbon}) generates
the run times for Gibbon and Marmoset, prints them in a tabular form,
and stores them in CSV files. In addition to the raw run times, the
script also prints out the speedups in a separate column (the last
column). In particular,

\begin{quote}
\texttt{speedup} = (slowest Gibbon, i.e, red color in the table) /
\texttt{Marmoset\_solver}.
\end{quote}

\hypertarget{approximate-timings}{%
\subsubsection{Approximate timings}\label{approximate-timings}}

\begin{enumerate}
\def\labelenumi{\arabic{enumi}.}
\tightlist
\item
  \texttt{\textasciitilde{}/vsGibbon/generate\_runtimes.py}

  \begin{itemize}
  \tightlist
  \item
    \texttt{small} mode: \textless10 minutes
  \item
    \texttt{vsSML} mode: \textless60 minutes
  \item
    default mode: \textless60 minutes
  \end{itemize}
\item
  \texttt{\textasciitilde{}/vsGibbon/generate\_compile\_times.py}

  \begin{itemize}
  \tightlist
  \item
    default mode: \textless10 minutes
  \end{itemize}
\item
  \texttt{\textasciitilde{}/vsGibbon/generate\_cache\_stats.py}

  \begin{itemize}
  \tightlist
  \item
    default mode: \textless30 minutes
  \end{itemize}
\item
  \texttt{\textasciitilde{}/vsSML/generate\_sml\_numbers.py}

  \begin{itemize}
  \tightlist
  \item
    default mode: \textasciitilde100 minutes
  \end{itemize}
\item
  \texttt{\textasciitilde{}/vsGHC/generate\_ghc\_numbers.py}

  \begin{itemize}
  \tightlist
  \item
    \texttt{small} mode: \textasciitilde15 minutes
  \item
    default mode: \textasciitilde100 minutes
  \end{itemize}
\end{enumerate}

\hypertarget{output-files}{%
\subsubsection{Output files}\label{output-files}}

\hypertarget{csv-files}{%
\paragraph{CSV files}\label{csv-files}}

\begin{enumerate}
\def\labelenumi{\arabic{enumi}.}
\tightlist
\item
  \texttt{\textasciitilde{}/vsGibbon/generate\_runtimes.py}

  \begin{itemize}
  \tightlist
  \item
    \texttt{vsSML} mode: .csv files are written to
    \texttt{\textasciitilde{}/vsGibbon/large\_sml} (for comparison to
    MLton)
  \item
    \texttt{small} mode: .csv files are written to
    \texttt{\textasciitilde{}/vsGibbon/small} (for comparison to GHC)
  \item
    default mode: .csv files are written to
    \texttt{\textasciitilde{}/vsGibbon/large} (for comparison to GHC)
  \end{itemize}
\item
  \texttt{\textasciitilde{}/vsGibbon/generate\_cache\_stats.py}

  \begin{itemize}
  \tightlist
  \item
    default mode: .csv files are written to
    \texttt{\textasciitilde{}/vsGibbon/large}. It outputs three .csv
    files for Table 5.
  \end{itemize}
\end{enumerate}

\hypertarget{pdf-files}{%
\paragraph{PDF files}\label{pdf-files}}

\begin{enumerate}
\def\labelenumi{\arabic{enumi}.}
\tightlist
\item
  \texttt{\textasciitilde{}/vsGibbon/generate\_compile\_times.py}
  outputs three PDF files:

  \begin{itemize}
  \tightlist
  \item
    \texttt{FilterBlogCompileTimes.pdf},
  \item
    \texttt{ContentSearchCompileTimes.pdf}, and
  \item
    \texttt{TagSearchCompileTimes.pdf} for the three compile time
    subfigures in Figure 9 respectively. The files will be written in
    the working directory.
  \end{itemize}
\item
  \texttt{\textasciitilde{}/vsSML/generate\_sml\_numbers.py} outputs
  three PDF files:

  \begin{itemize}
  \tightlist
  \item
    \texttt{SpeedupMarmosetSmlFilterBlogs.pdf},
  \item
    \texttt{SpeedupMarmosetSmlContentSearch.pdf}, and
  \item
    \texttt{SpeedupMarmosetSmlTagSearch.pdf}. The files will be written
    in \texttt{\textasciitilde{}/vsSML/large}.
  \end{itemize}
\item
  \texttt{\textasciitilde{}/vsGHC/generate\_ghc\_numbers.py} outputs
  three PDF files:

  \begin{itemize}
  \tightlist
  \item
    \texttt{SpeedupMarmosetGhcFilterBlogs.pdf},
  \item
    \texttt{SpeedupMarmosetGhcContentSearch.pdf}, and
  \item
    \texttt{SpeedupMarmosetGhcTagSearch.pdf}. The files will be written
    in \texttt{\textasciitilde{}/vsGHC/large} or
    \texttt{\textasciitilde{}/vsGHC/small} depending on which mode we
    run in.
  \end{itemize}
\end{enumerate}

\hypertarget{mapping-of-csv-files-to-the-tables-in-the-paper}{%
\paragraph{Mapping of CSV files to the tables in the
paper}\label{mapping-of-csv-files-to-the-tables-in-the-paper}}

For every table, we show below how the row and column names in the paper
map on the CSV files and on the filenames of the benchmarks in the
artifact. The filenames are either for source files (if compiled with
Gibbon) or for binaries produced by Marmoset from one of the versions of
the source files (the particular version can be decoded from the binary
name, e.g.: \texttt{layout2ListLenGreedy} is compiled from
\texttt{layout2ListLen.hs}).

Generally, the names of CSV files do not much the table numbers in the
paper (e.g.~the \texttt{ListLength} numbers are reported in the
\texttt{Table2.csv} file, while in the paper these numbers are a part of
Table 1), but the mapping below resolves the mismatches. Another
difference is that our scripts print the median times along with the
mean and lower and upper bounds of the confidence interval that are
reported in the paper.

\begin{itemize}
\item
  Table 1

  \begin{itemize}
  \tightlist
  \item
    Line \texttt{ListLength} maps to \texttt{Table2.csv}

    \begin{itemize}
    \tightlist
    \item
      \texttt{List} maps to \texttt{layout2ListLen.hs}
    \item
      \texttt{List\textquotesingle{}} maps to \texttt{layout1ListLen.hs}
    \item
      \texttt{M\_Greedy} maps to \texttt{layout2ListLenGreedy}
    \item
      \texttt{M\_Solver} maps to \texttt{layout2ListLenSolver}
    \end{itemize}
  \item
    Line \texttt{LogicEval} maps to \texttt{Table3.csv}

    \begin{itemize}
    \tightlist
    \item
      \texttt{lr} maps to \texttt{eval\_l.hs}
    \item
      \texttt{rl} maps to \texttt{eval\_r.hs}
    \item
      \texttt{M\_Greedy} maps to \texttt{eval\_rGreedy}
    \item
      \texttt{M\_Solver} maps to \texttt{eval\_rSolver}
    \end{itemize}
  \item
    Line \texttt{Rightmost}: \texttt{Table5.csv}

    \begin{itemize}
    \tightlist
    \item
      \texttt{lr} maps to \texttt{TreeRightMost\_l.hs}
    \item
      \texttt{rl} maps to \texttt{TreeRightMost\_r.hs}
    \item
      \texttt{M\_Greedy} maps to \texttt{TreeRightMost\_lGreedy}
    \item
      \texttt{M\_Solver} maps to \texttt{TreeRightMost\_lSolver}
    \end{itemize}
  \end{itemize}
\item
  Table 2

  \begin{itemize}
  \tightlist
  \item
    Line \texttt{AddOneTree}: \texttt{Table4a.csv}

    \begin{itemize}
    \tightlist
    \item
      \texttt{Misalgn\_pre} maps to \texttt{TreeAddOnePrePost.hs}
    \item
      \texttt{Algn\_pre} maps to \texttt{TreeAddOnePre.hs}
    \item
      \texttt{Algn\_in} maps to \texttt{TreeAddOneIn.hs}
    \item
      \texttt{Algn\_post} maps to \texttt{TreeAddOnePost.hs}
    \item
      \texttt{M\_Greedy} maps to \texttt{TreeAddOnePreGreedy}
    \item
      \texttt{M\_Solver} maps to \texttt{TreeAddOnePreSolver}
    \end{itemize}
  \item
    Line \texttt{ExpTree}: \texttt{Table4b.csv}

    \begin{itemize}
    \tightlist
    \item
      \texttt{Misalgn\_pre} maps to \texttt{TreeExpoPrePost.hs}
    \item
      \texttt{Algn\_pre} maps to \texttt{TreeExpoPre.hs}
    \item
      \texttt{Algn\_in} maps to \texttt{TreeExpoIn.hs}
    \item
      \texttt{Algn\_post} maps to \texttt{TreeExpoPost.hs}
    \item
      \texttt{M\_Greedy} maps to \texttt{TreeExpoPreGreedy}
    \item
      \texttt{M\_Solver} maps to \texttt{TreeExpoPreSolver}
    \end{itemize}
  \item
    Line \texttt{CopyTree}: \texttt{Table4c.csv}

    \begin{itemize}
    \tightlist
    \item
      \texttt{Misalgn\_pre} maps to \texttt{TreeCopyPrePost.hs}
    \item
      \texttt{Algn\_pre} maps to \texttt{TreeCopyPre.hs}
    \item
      \texttt{Algn\_in} maps to \texttt{TreeCopyIn.hs}
    \item
      \texttt{Algn\_post} maps to \texttt{TreeCopyPost.hs}
    \item
      \texttt{M\_Greedy} maps to \texttt{TreeCopyPreGreedy}
    \item
      \texttt{M\_Solver} maps to \texttt{TreeCopyPreSolver}
    \end{itemize}
  \end{itemize}
\item
  Table 3

  \begin{itemize}
  \tightlist
  \item
    Line \texttt{FilterBlogs}: \texttt{Table6a.csv}

    \begin{itemize}
    \tightlist
    \item
      \texttt{hiadctb} maps to \texttt{layout1FilterBlogs.hs}
    \item
      \texttt{ctbhiad} maps to \texttt{layout2FilterBlogs.hs}
    \item
      \texttt{tbchiad} maps to \texttt{layout3FilterBlogs.hs}
    \item
      \texttt{tcbhiad} maps to \texttt{layout4FilterBlogs.hs}
    \item
      \texttt{btchiad} maps to \texttt{layout5FilterBlogs.hs}
    \item
      \texttt{bchiadt} maps to \texttt{layout7FilterBlogs.hs}
    \item
      \texttt{cbiadht} maps to \texttt{layout8FilterBlogs.hs}
    \item
      \texttt{M\_Greedy} maps to \texttt{layout8FilterBlogsGreedy}
    \item
      \texttt{M\_Solver} maps to \texttt{layout8FilterBlogsSolver}
    \end{itemize}
  \item
    Line \texttt{EmphContent}: \texttt{Table6b.csv}

    \begin{itemize}
    \tightlist
    \item
      \texttt{hiadctb} maps to \texttt{layout1ContentSearch.hs}
    \item
      \texttt{ctbhiad} maps to \texttt{layout2ContentSearch.hs}
    \item
      \texttt{tbchiad} maps to \texttt{layout3ContentSearch.hs}
    \item
      \texttt{tcbhiad} maps to \texttt{layout4ContentSearch.hs}
    \item
      \texttt{btchiad} maps to \texttt{layout5ContentSearch.hs}
    \item
      \texttt{bchiadt} maps to \texttt{layout7ContentSearch.hs}
    \item
      \texttt{cbiadht} maps to \texttt{layout8ContentSearch.hs}
    \item
      \texttt{M\_Greedy} maps to \texttt{layout8ContentSearchGreedy}
    \item
      \texttt{M\_Solver} maps to \texttt{layout8ContentSearchSolver}
    \end{itemize}
  \item
    Line \texttt{TagSearch}: \texttt{Table6c.csv}

    \begin{itemize}
    \tightlist
    \item
      \texttt{hiadctb} maps to \texttt{layout1TagSearch.hs}
    \item
      \texttt{ctbhiad} maps to \texttt{layout2TagSearch.hs}
    \item
      \texttt{tbchiad} maps to \texttt{layout3TagSearch.hs}
    \item
      \texttt{tcbhiad} maps to \texttt{layout4TagSearch.hs}
    \item
      \texttt{btchiad} maps to \texttt{layout5TagSearch.hs}
    \item
      \texttt{bchiadt} maps to \texttt{layout7TagSearch.hs}
    \item
      \texttt{cbiadht} maps to \texttt{layout8TagSearch.hs}
    \item
      \texttt{M\_Greedy} maps to \texttt{layout8TagSearchGreedy}
    \item
      \texttt{M\_Solver} maps to \texttt{layout8TagSearchSolver}
    \end{itemize}
  \end{itemize}
\item
  Table 4

  \begin{itemize}
  \tightlist
  \item
    Line \texttt{FilterBlogs}: \texttt{Table7a.csv}

    \begin{itemize}
    \tightlist
    \item
      \texttt{Gibbon} maps to \texttt{manyFuncs-FilterBlogs}
    \item
      \texttt{M\_Greedy} maps to \texttt{manyFuncsGreedy-FilterBlogs}
    \item
      \texttt{M\_Solver} maps to \texttt{manyFuncsSolver-FilterBlogs}
    \end{itemize}
  \item
    Line \texttt{EmphContent}: \texttt{Table7b.csv}

    \begin{itemize}
    \tightlist
    \item
      \texttt{Gibbon} maps to \texttt{manyFuncs-EmphKeyword}
    \item
      \texttt{M\_Greedy} maps to \texttt{manyFuncsGreedy-EmphKeyword}
    \item
      \texttt{M\_Solver} maps to \texttt{manyFuncsSolver-EmphKeyword}
    \end{itemize}
  \item
    Line \texttt{TagSearch}: \texttt{Table7c.csv}

    \begin{itemize}
    \tightlist
    \item
      \texttt{Gibbon} maps to \texttt{manyFuncs-EmphKeywordInTag}
    \item
      \texttt{M\_Greedy} maps to
      \texttt{manyFuncsGreedy-EmphKeywordInTag}
    \item
      \texttt{M\_Solver} maps to
      \texttt{manyFuncsSolver-EmphKeywordInTag}
    \end{itemize}
  \end{itemize}
\item
  Table 5

  \begin{itemize}
  \tightlist
  \item
    Line \texttt{FilterBlogs}: \texttt{Table8a.csv} (if run outside of
    the docker using the cache script)

    \begin{itemize}
    \tightlist
    \item
      \texttt{hiadctb} maps to \texttt{layout1FilterBlogs.hs}
    \item
      \texttt{ctbhiad} maps to \texttt{layout2FilterBlogs.hs}
    \item
      \texttt{tbchiad} maps to \texttt{layout3FilterBlogs.hs}
    \item
      \texttt{tcbhiad} maps to \texttt{layout4FilterBlogs.hs}
    \item
      \texttt{btchiad} maps to \texttt{layout5FilterBlogs.hs}
    \item
      \texttt{bchiadt} maps to \texttt{layout7FilterBlogs.hs}
    \item
      \texttt{cbiadht} maps to \texttt{layout8FilterBlogs.hs}
    \item
      \texttt{M\_Greedy} maps to \texttt{layout8FilterBlogsGreedy}
    \item
      \texttt{M\_Solver} maps to \texttt{layout8FilterBlogsSolver}
    \end{itemize}
  \item
    Line \texttt{EmphContent}: \texttt{Table8b.csv}

    \begin{itemize}
    \tightlist
    \item
      \texttt{hiadctb} maps to \texttt{layout1ContentSearch.hs}
    \item
      \texttt{ctbhiad} maps to \texttt{layout2ContentSearch.hs}
    \item
      \texttt{tbchiad} maps to \texttt{layout3ContentSearch.hs}
    \item
      \texttt{tcbhiad} maps to \texttt{layout4ContentSearch.hs}
    \item
      \texttt{btchiad} maps to \texttt{layout5ContentSearch.hs}
    \item
      \texttt{bchiadt} maps to \texttt{layout7ContentSearch.hs}
    \item
      \texttt{cbiadht} maps to \texttt{layout8ContentSearch.hs}
    \item
      \texttt{M\_Greedy} maps to \texttt{layout8ContentSearchGreedy}
    \item
      \texttt{M\_Solver} maps to \texttt{layout8ContentSearchSolver}
    \end{itemize}
  \item
    Line \texttt{TagSearch}: \texttt{Table8c.csv}

    \begin{itemize}
    \tightlist
    \item
      \texttt{hiadctb} maps to \texttt{layout1TagSearch.hs}
    \item
      \texttt{ctbhiad} maps to \texttt{layout2TagSearch.hs}
    \item
      \texttt{tbchiad} maps to \texttt{layout3TagSearch.hs}
    \item
      \texttt{tcbhiad} maps to \texttt{layout4TagSearch.hs}
    \item
      \texttt{btchiad} maps to \texttt{layout5TagSearch.hs}
    \item
      \texttt{bchiadt} maps to \texttt{layout7TagSearch.hs}
    \item
      \texttt{cbiadht} maps to \texttt{layout8TagSearch.hs}
    \item
      \texttt{M\_Greedy} maps to \texttt{layout8TagSearchGreedy}
    \item
      \texttt{M\_Solver} maps to \texttt{layout8TagSearchSolver}
    \end{itemize}
  \end{itemize}
\end{itemize}

\hypertarget{miscellaneous}{%
\subsubsection{Miscellaneous}\label{miscellaneous}}

\begin{itemize}
\item
  The output from
  \texttt{\textasciitilde{}/vsGibbon/generate\_runtimes.py} is written
  to CSV files and stdout. However, the output written to stdout may be
  compressed (\texttt{...} between columns in the tables means
  compressed output). The user may try to decrease the font size in the
  terminal to see the full output. In any case, the CSV files have the
  full output.
\item
  Some scripts output PDF files, which can be transferred out of the
  container using
  \href{https://docs.docker.com/engine/reference/commandline/container_cp/}{\texttt{docker\ cp}}
  command, in order to view them.
\item The following scripts can take an additional \texttt{-\/-verbose} flag to show extra output while the
script is running.
  \item \texttt{\textasciitilde{}/vsGibbon/generate\_runtimes.py}
  \item \texttt{\textasciitilde{}/vsGHC/generate\_sml\_numbers.py}
  \item \texttt{\textasciitilde{}/vsGHC/generate\_ghc\_numbers.py} 
  
\end{itemize}

\hypertarget{build-marmoset-and-papi-outside-docker-for-generating-table-5}{%
\subsubsection{Build Marmoset and PAPI outside Docker for generating
Table
5}\label{build-marmoset-and-papi-outside-docker-for-generating-table-5}}

\textbf{Install dependencies to build Marmoset on Ubunutu 22.04:}
\begin{verbatim}
 $ sudo apt-get update 
 $ sudo apt-get install software-properties-common \
                        libgc-dev \
                        libgmp-dev \
                        build-essential \
                        uthash-dev \
                        vim wget curl
\end{verbatim}

\textbf{Install Racket:}
\begin{verbatim}
 $ wget --no-check-certificate \
         https://mirror.racket-lang.org/installers/7.5/racket-7.5-x86_64-linux.sh
 $ chmod +x racket-7.5-x86_64-linux.sh
 $ ./racket-7.5-x86_64-linux.sh
\end{verbatim}

\textbf{Install the Haskell toolchain:}
\begin{verbatim}
 $ curl --proto '=https' --tlsv1.2 -sSf https://get-ghcup.haskell.org \ 
    | BOOTSTRAP_HASKELL_NONINTERACTIVE=1 \
    BOOTSTRAP_HASKELL_GHC_VERSION=9.4.6 \ 
    BOOTSTRAP_HASKELL_CABAL_VERSION=3.8.1.0 \ 
    BOOTSTRAP_HASKELL_INSTALL_STACK=1 \ 
    BOOTSTRAP_HASKELL_INSTALL_HLS=1 \
    BOOTSTRAP_HASKELL_ADJUST_BASHRC=P sh
\end{verbatim}

\textbf{Install the Rust toolchain:}
\begin{verbatim}
 $ curl --proto '=https' --tlsv1.2 -sSf https://sh.rustup.rs \
        | sh -s -- -y --default-toolchain=1.71.0
\end{verbatim}

\textbf{Build Marmoset:}
\begin{verbatim}
$ tar -xf marmoset.tar
$ cd marmoset && source set_env.sh
$ cd gibbon-compiler && cabal v2-build exe:gibbon && cabal v2-install exe:gibbon
\end{verbatim}

\textbf{Install PAPI:}
\begin{verbatim}
$ wget https://github.com/icl-utk-edu/papi/archive/refs/tags/papi-7-1-0-t.tar.gz && \
    mkdir papi && \
    tar -xvzf papi-7-1-0-t.tar.gz -C papi && \
    cd papi && cd papi-papi-7-1-0-t && cd src && \
    ./configure && make -j10 && make install
$ export PAPI_EVENTS="PAPI_TOT_INS,PAPI_TOT_CYC,PAPI_L2_DCM"
\end{verbatim}

\begin{itemize}
\tightlist
\item
  Run \texttt{vsGibbon/generate\_cache\_stats.py} to generate Table 8 of
  the paper.
\end{itemize}

\hypertarget{for-authors-claiming-a-reusable-badge}{%
\subsection{Proof of a reusable badge}\label{for-authors-claiming-a-reusable-badge}}

The implementation of Marmoset is currently under review for inclusion
in the main line of the Gibbon project, which is open source. The way to
build our code is shown in the Dockerfile, and it is identical to the
standard instructions to build Gibbon. The benchmarks are available as
the part of this artifact.

\hypertarget{structure-of-the-marmoset-implementation}{%
\subsubsection{Structure of the Marmoset
implementation}\label{structure-of-the-marmoset-implementation}}

Marmoset is an extension to Gibbon, an open source compiler written in
Haskell. Gibbon compiles high level programs written in a subset of
Haskell to operate on serialized data in memory. The compiler is written
as a series of micro passes that do a small amount of work. Marmoset is
implemented as a combination of passes in that pipeline. It is
straightforward to register and write a pass in the compiler. Hence our
framework can be extended with more complex optimizations without
substantial changes to the compiler. This makes future research easy to
build on top of the current framework.

The solver used in Marmoset is open source, and the solver can be
changed for other solvers.

Marmoset consists of the following modules extending Gibbon (files
locations relative to
\texttt{\textasciitilde{}/marmoset/gibbon-compiler/src} in the
container):

\begin{itemize}
\item
  \texttt{Gibbon/Passes/ControlFlowGraph.hs} --- This pass adds a static
  analysis to generate the control flow graph of the functions in the
  program.
\item
  \texttt{Gibbon/Passes/DefinitionUseChains.hs} --- This pass does a
  def-use, use-def chains analysis for each function.
\item
  \texttt{Gibbon/Passes/CallGraph.hs} --- This pass generates the call
  graph from the program.
\item
  \texttt{Gibbon/Passes/AccessPatternsAnalysis.hs} --- This pass
  generates a graph recording the access patterns between fields of a
  data constructor for each function in the program.
\item
  \texttt{Gibbon/Passes/SolveLayoutConstrs.hs} --- This pass generates
  the ILP constraints and calls the solver.
\item
  \texttt{Gibbon/Passes/OptimizeADTLayout.hs} --- This pass optimizes
  the layout of each data constructor globally.
\end{itemize}


% ARTIFACT: optional appendix
%\appendix
%\section{Morbi eros magna}

%Morbi eros magna, vestibulum non posuere non, porta eu quam. Maecenas vitae orci risus, eget imperdiet mauris. Donec massa mauris, pellentesque vel lobortis eu, molestie ac turpis. Sed condimentum convallis dolor, a dignissim est ultrices eu. Donec consectetur volutpat eros, et ornare dui ultricies id. Vivamus eu augue eget dolor euismod ultrices et sit amet nisi. Vivamus malesuada leo ac leo ullamcorper tempor. Donec justo mi, tempor vitae aliquet non, faucibus eu lacus. Donec dictum gravida neque, non porta turpis imperdiet eget. Curabitur quis euismod ligula \cite{DBLP:books/mk/GrayR93,DBLP:conf/focs/FOCS16,DBLP:conf/focs/HopcroftPV75,DBLP:journals/cacm/Dijkstra68a,DBLP:journals/cacm/Knuth74}. 


% ARTIFACT: include here any additional references, if needed...

%%
%% Bibliography
%%

%% Either use bibtex (recommended), 

\bibliography{darts-v2021-sample-article}

%% .. or use the thebibliography environment explicitely



\end{document}
